\documentclass[11pt, oneside]{article}   	% use "amsart" instead of "article" for AMSLaTeX format
\usepackage{geometry}                		% See geometry.pdf to learn the layout options. There are lots.
\geometry{letterpaper}                   		% ... or a4paper or a5paper or ... 
%\geometry{landscape}                		% Activate for rotated page geometry
%\usepackage[parfill]{parskip}    		% Activate to begin paragraphs with an empty line rather than an indent
\usepackage{graphicx}				% Use pdf, png, jpg, or eps§ with pdflatex; use eps in DVI mode
								% TeX will automatically convert eps --> pdf in pdflatex		
\usepackage{amssymb}

%SetFonts
\usepackage{subfig}
\usepackage{algorithm}
\usepackage{algpseudocode}
\usepackage{algorithmicx}
%\usepackage{algorithmic}
\usepackage{comment}
\usepackage{amsmath}
%SetFonts

\newcommand{\td}[2]{\frac{\diff #1}{\diff #2}}
\newcommand{\pd}[2]{\frac{{\partial}#1}{{\partial}#2}}
\newcommand{\pdp}[3]{\frac{{\partial^{#3}}#1}{{\partial}#2^{#3}}}
\newcommand{\pdd}[2]{\frac{\partial^2#1}{\partial#2^2}}
\newcommand{\mb}[1]{\mathbf{#1}}
\newcommand{\mbb}[1]{\mathbb{#1}}
\newcommand{\mc}[1]{\mathcal{#1}}
\newcommand{\nor}[1]{\left\| #1 \right\|}
\newcommand{\snor}[1]{\left| #1 \right|}
\newcommand{\LRp}[1]{\left( #1 \right)}
\newcommand{\LRs}[1]{\left[ #1 \right]}
\newcommand{\LRa}[1]{\left\langle #1 \right\rangle}
\newcommand{\LRb}[1]{\left| #1 \right|}
\newcommand{\LRc}[1]{\left\{ #1 \right\}}
\newcommand{\LRlim}[1]{\left.{ #1 }\right|}

\title{Brief Article}
\author{The Author}
%\date{}							% Activate to display a given date or no date

\begin{document}
\maketitle
%\section{}
%\subsection{}

\begin{algorithm}
\begin{algorithmic}[1]
\Procedure{Time Stepping Method}{}
\State Solve for the max flux over the time step and attain the end of step fluid mass $m^{n}$ and end of step 
pressure $p^{n}$, 
\begin{itemize}
\item $m^{n+1} = m^{n} + k^{n} \rho^{n} \left( \nabla p^{n} \right) \Delta t$
\end{itemize}

\State Update the velocity to the mid-step 
\begin{itemize}
\item $\mathbf{v}^{n+\frac{1}{2}} = \mathbf{v}^{n} + \mathbf{a}^{n} \frac{\Delta t}{2}$
\end{itemize}

\State Update the displacement to the end of the step
\begin{itemize}
\item $\mathbf{x}^{n+1} = \mathbf{x}^{n} + \mathbf{v}^{n+\frac{1}{2}} \Delta t$
\end{itemize}

\State Calculate deformation input to constitutive model and update material state to $Q^{n+1}$

\State Solve for the acceleration $t^{n+1} \left( a_i^{n+1} \right)$. Note that this includes the fluid 
pressure $p^{n+1}$ applied as a boundary condition

\State Update the velocity to the end of step 

\begin{itemize}
\item $\mathbf{v}^{n+1} = \mathbf{v}^{n+\frac{1}{2}} + \mathbf{a}^{n+1} \frac{\Delta t}{2} $ 
\end{itemize}


\EndProcedure
\end{algorithmic}
\caption{Calculate deformation input to constitutive model and update material state to $Q^{n}$}
\end{algorithm}

\begin{algorithm}
\begin{algorithmic}[1]
\Procedure{Calculate deformation input to constitutive model and update material state to $Q^{n}$}{}
\For{Each Element}
\State Gather nodal nodal degrees of freedom for total/inc displacement
\For{Each quadrature point}
\State Calculate gradient for total/inc displacement (mult-mult operations)
\State Calculate gradient for the force (local operations with $3 \times 3$ matrices)
\State Incremental kinematic step
\State Constitutive update step
\State Accumulate quadrature sum
\EndFor
\State Scatter nodal degrees of freedom to force vector
\EndFor
\EndProcedure
\end{algorithmic}
\caption{Calculate deformation input to constitutive model and update material state to $Q^{n}$}
\end{algorithm}




\end{document}  